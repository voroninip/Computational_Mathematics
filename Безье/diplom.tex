\documentclass[a4paper,12pt]{report}
\usepackage[T2A]{fontenc}
\usepackage[utf8]{inputenc}
\usepackage[russian]{babel}
\usepackage{hyperref}
\usepackage{graphicx}

\usepackage{makecell}
\usepackage{multirow}

\usepackage{indentfirst}

\setcounter{tocdepth}{2}

\renewcommand{\rmdefault}{ftm}
\renewcommand\theadfont{\normalsize}

\usepackage{geometry}
\geometry{a4paper}
\geometry{left=35mm,right=15mm,top=20mm,bottom=20mm}
\geometry{headheight=2ex,headsep=10mm,footskip=10mm}

\renewcommand{\maketitle}{
\begin{titlepage}
	\begin{center}
           	МИНИСТЕРСТВО ОБРАЗОВАНИЯ РОССИЙСКОЙ ФЕДЕРАЦИИ\par
		МОСКОВСКИЙ ФИЗИКО-ТЕХНИЧЕСКИЙ ИНСТИТУТ (ГОСУДАРСТВЕННЫЙ УНИВЕРСИТЕТ)\par
		\hrulefill\\
		ФАКУЛЬТЕТ УПРАВЛЕНИЯ И ПРИКЛАДНОЙ МАТЕМАТИКИ\par
		КАФЕДРА ВЫЧИСЛИТЕЛЬНОЙ МАТЕМАТИКИ\par
		\vspace{60mm}
	{\LARGE \bf{Раскладка графа на плоскости с учетом локализации его вершин в двумерных односвязных областях}}\par
		\vspace{20mm}
		{\LargeВыпускная квалификационная работа на степень бакалавра}\par
		{\Largeстудента 571 группы ФУПМ Лейриха И.В.}\par
	\end{center}

	\vfill
        \hfill
        \begin{minipage}[t]{80mm}\flushleft

            Научный руководитель \par
            к.ф.-м.н. Симаков С.С.\par

        \end{minipage}
        \vspace{10mm}
        \vfill



	\begin{center}
		Москва 2009
	\end{center}

\end{titlepage}
\stepcounter{page}
}

\linespread{1.3}
\graphicspath{{images/}}

\begin{document}

\maketitle
\tableofcontents

\chapter*{Введение}
\addcontentsline{toc}{chapter}{Введение}

Задача визуализации сложных структур возникает во многих сферах науки, техники и бизнеса. Для наглядного представления информации часто используют графы. Графы идеально подходят для отображения любой информации, представимой в виде объектов и связей между ними.  Разработка программного обеспечения, веб-дизайн и оптимизация сайтов, анализ и администрирование компьютерных сетей, бизнес-менеджмент,  медицина, химия, генеалогия - вот далеко неполный список областей их применения.

В связи с этим несомненно важным становится разработка программных продуктов, предназначенных для создания и редактирования графов. Данные программы нуждаются  в алгоритмах раскладки графов, позволяющих придать графам эстетический вид,  упростить и облегчить восприятие и анализ информации пользователями. Теория раскладки графов, являющаяся частью теории графов, опираясь на аналитическую геометрию, топологию позволяет получить изображения графов, удовлетворяющие предъявляемым требованиям. Этот раздел математики быстро развивается \cite{graphdrawing},\cite{gdea}. Разработаны различные алгоритмы раскладки графов, такие как: спектральная, древовидная, ортогональная, иерархическая, симметрическая раскладки, а также огромное количество вариаций алгоритмов, основанных на задании сил притяжения-отталкивания между вершинами графа (force-directed algorithms, далее DF алгоритм). Выбор стратегии зависит от конкретной задачи, однако последние являться наиболее распространенными благодаря своей простоте, теоретической обоснованности, и хорошим результатам, которые они дают. Помимо этого необходимо разрабатывать надстройки на основной алгоритм, позволяющие учитывать формы вершин, привязку отдельных вершин к некоторым областям пространства, учет существования подписей у вершин и многие другие ограничения, возникающие при решении конкретных задач. 

Во многих задачах вершины могут быть представлены в виде точек или простых геометрических фигур, таких как прямоугольник или эллипс. Однако, существуют задачи где форма вершин представляет собой сложную фигуру, и представление ее в виде более простой геометрической фигуры нежелательно или принципиально невозможно (Рис.~\ref{ris:network}). Раскладка графа в пространстве при помощи DF алгоритма – итерационный  процесс, с большим количеством итеаций. Учет сложной формой вершин и локаций потребует проверки отсутствия взаимного пересечения для каждой пары вершин; ровно как  проверки правильной локализации каждой отдельно взятой вершины на каждом шаге работы алгоритма.  Подобные операции с  вершинами, формы которых заданы растровыми изображениями, требуют значительных затрат времени и системных ресурсов.  Аппроксимация границ областей, занимаемых вершинами, а также границ областей локализации вершин при помощи полиномиальных кривых и, в частности кривых Безье, позволит значительно сократить время работы и уменьшить объем используемой памяти.

\begin{figure}[h]
\center{\includegraphics[width=0.8\linewidth]{network}}
\caption{Модель электрических сетей юго-восточной Европпы.}
\label{ris:network}
\end{figure}

Существует множество программных реализаций алгоритмов раскладки как коммерческого характера так и opensorce проектов, например \cite{softwarelist}. Однако в них не реализована работа с вершинами сложной формы. Поэтому эта задача является актуальной. Целью работы является разработка методов учета разнообразных форм вершин и локализаций в алгоритмах раскладки графов, с использованием преимуществ векторной графики. Их реализация и интеграция в  графопостроитель Esher. Предполагается рассмотреть все этапы решения поставленной задачи, такие как: распознание формы области, заданной растровым изображением, аппроксимация границы векторными кривыми, разработка нового формата хранения форм, и алгоритмы работы с ним.

\chapter{Представление областей в виде списка кривых Безье (Б-области)}

Сформулируем задачи. Имеется растровое изображение, например файл в формате .png или .jpg. Форма вершины или области локализации хранится в нем в виде связной области черного цвета на белом фоне. Необходимо преобразовать его в некоторое множество векторных объектов, несущих информацию о форме объекта. А также, научиться выполнять простейшие операции, например: определять лежит ли данная точка внутри области, заданной таким образом; находить проекцию точки на область. 

В нашей работе мы будем использовать параметризованные кривые, в которых обе координаты имеют полиномиальную зависимость от параметра. При выборе порядка полинома будем учитывать, что полиномы низкого порядка могут не давать достаточной степени аппроксимации, в свою очередь высокий порядок требует больших системных затрат. По этим и некоторым другим причинам обычно останавливаются на кривых третьего порядка. Будем представлять их в базисе из полиномов Бернштейна третьего порядка ${b}_{i,3}(t)={3\choose i} t^i (1-t)^{3-i},\quad i=0,\ldots 4$ 

\begin{equation} \label{eq:bernstein} 
\overrightarrow{B}\left(t\right)=b_{0,3}*\overrightarrow{P_0}+b_{1,3}*\overrightarrow{P_1}+b_{2,3}*\overrightarrow{P_2}{+b}_{3,3}*\overrightarrow{P_3},       t\in [0,1] 
\end{equation}

Представление в базисе из полиномов Берштейна удобно с той точки зрения что опорные точки имеют физический смысл. Так $\overrightarrow{P_0}$ и  $\overrightarrow{P_3}$  векторы начальной и конечной точек прямой. Векторы  $\overrightarrow{P_0P_1}\ $ $\overrightarrow{P_0P_1}$ совпадают по направлению с касательными к кривой на концах а их модуль пропорционален близости кривой к этим касательным.  

Записанные в таком виде кривые называются кривыми Безье третьего порядка~\cite{bezier}. Изобретенные в середине прошлого века они приобрели значительную популярность благодаря простоте хранения и редактирования. Так называемые "Пути", широко используемые в таких графических редакторах как Inkscape, Adobe Illustrator, Adobe Photoshop и GIMP, являются последовательностями Безье-кривых, соединенных вместе. Кривые Безье так же используются при создании анимации в приложениях Adobe Flash, Adobe After Effects, Microsoft Expression Blend, Blender, и Autodesk 3ds max.

\begin{figure}[h]
\center{\includegraphics[width=0.7\linewidth]{bezier}}
\caption{Кривая Безье третьего порядка.}
\label{ris:bezier}
\end{figure}

Таким образом использование кривых Безье основано на целом ряде причин, из которых особо следует выделить, не требовательность к ресурсам, возможность при желании редактировать границу области по окончанию работы алгоритма аппроксимации, совместимость с форматом популярного редактора Inkscape. Область, представленную в виде списка кривых Безье будем называть Б-областью. 

Теперь приступим к решению поставленных задач.

\section{Распознание граничных точек} 
Перед тем как приступить непосредственно к аппроксимации необходимо получить входные данные из растрового изображения. В нашем случае в роли таких данных будет выступать список из координат подарят идущих граничных точек. Граничной точкой будем называть точку, принадлежащую области и имеющую по крайней мере одну соседнюю точку не принадлежащую области. Для нахождения этого списка в случае односвязной области достаточно выполнить следующие действия  
 
\begin{enumerate}
\item находим одну произвольную граничную точку — она будет первая в нашем списке. Можно реализовать, перебирая все точек изображения начиная сверху, пока не встретим граничную;
\item выбираем соседнюю граничную точку, которая еще не была занесена в список;
\item повторяем второе действие пока возможно.
\end{enumerate} 

Односвязность области обеспечивает окончание работы алгоритма в тот момент когда вся граница пройдена. Полученный список далее будет преобразован в Б-область
 
\begin{figure}[h]
\center{\includegraphics[width=0.7\linewidth]{border}}
\caption{Обход границы.}
\label{ris:border}
\end{figure}

\section{Аппроксимация границы параметрически заданными кривыми}

\subsection{Разложения по полиномам Чебышева}\label{ss:Chebaprox}

В теории аппроксимации часто используются полиномы Чебышева. Существует много работ связанных с ними и их применением \cite{chislmetod}, \cite{numrical analysis}. Чтобы воспользоваться некоторыми полезными свойствами полиномов Чебышева, запишем выражение для кривой третьего порядка в базисе из полиномов Чебышева: 

\begin{equation} \label{eq:cheb} 
\overrightarrow{B}\left(z\right)=\sum^3_{i=0}{T_i(z)\overrightarrow{a_i}},      z\in [-1,1]
\end{equation}
где $\overrightarrow{a_i}$ - некоторые векторные коэффициенты, $T_i(z)$- полиномы Чебышева 
\[T_0\left(z\right)=1,\] 
\[T_1\left(z\right)=z,\] 
\[T_2\left(z\right)=2z^2-1,\] 
\[T_3\left(z\right)=4z^3-3z.\] 

Допустим мы мы хотим аппроксимировать ${\rm M}$ подряд идущих граничных точек параметрической кривой. Задача состоит в нахождении коэффициентов $\overrightarrow{a_i}$. Выбираем список координат точек данного элемента границы из полученного в предыдущем пункте списка для границы целиком.
Рассмотрим следующее определение~\cite{shokurov}. Чебышевской системой узлов порядка $n$ на отрезке $I=\in[a,b]$ называется система точек

\begin{equation} \label{eq:cheb node sist} 
t_k=(t_{1,n},t_{2,n},t_{3,n},..,t_{1,n})
\end{equation} 

\[t_{n,k}(I)=\frac{b-a}{2}\cos{\frac{(2k-1)\pi}{2n}+\frac{b+a}{2}}, n\geq0, 1\leq{k}\leq{n}\] 

Эти точки распределены по отрезку $I$ пропорционально распределению корней полинома Чебышева n-ого порядка на отрезке $[-1,1]$. Для дискретного случая списка, по аналогии построим систему узлов для аппроксимации по формуле:

\begin{equation} \label{eq:dis cheb node sist} 
t^*_k=(t^*_{1,n},t^*_{2,n},t^*_{3,n},..,,t^*_{1,n})
\end{equation} 

\[t^*_{n,k}=\lfloor\frac{M}{2}\cos{\frac{(2k-1)\pi}{2n}+\frac{M}{2}}\rfloor\] 

Взяв из списка координаты точек в узлах, выбранных таким образом, получаем два столбца высоты $N$: $\overrightarrow{x}$, $\overrightarrow{y}$. Так как аппроксимирующий сплайн задается двумя, параметрически заданными функциями $x(t)$ и $y(t)$, которые вычисляются независимо одним и тем же образом; далее будем рассматривать только $x$ координаты.

Подставляем пары параметр-значение из столбцов $\overrightarrow{r}$ и $\overrightarrow{x}$ в \ref{eq:cheb}, предполагая что прямая должна проходить рядом с настоящей границей в этой точке. Получаем переопределенную систему из $n$ уравнений с 4 переменными.
 
\begin{equation} \label{eq:system} 
x_j= \sum^3_{i=0}{T_i(r_j)a_i},       j=\overline{1,N}
\end{equation}

$r_j\in [-1,1]$ - корни многочлена Чебышева n-ого  порядка,

$a_i$ - искомые коэффициенты разложения.

Данная система в общем случае несовместная. Это эквивалентно проведению кубической кривой через $n>4$ опорных точек, взятых на, вообще говоря, произвольной гладкой кривой. Чтобы получить совместную систему относительно $a_i$, и в то же время сохранить больше информации о начальной кривой, воспользуемся леммой \cite{chislmetod}.

\vspace{0.3cm}

Лемма 1: Пусть $N$ - некоторое фиксированное число. Векторы, образованные значениями многочленов ${{\rm T}}_{{\rm n}}\left({\rm z}\right){\rm ,\ }{\rm n}{\rm <}{\rm N}$ в нулях  ${{\rm T}}_{{\rm N}}\left({\rm z}\right)$ образуют некоторую ортогональную систему, а именно 

\begin{equation} \label{eq:lemma} 
\frac{{\rm 2}}{{\rm N}}\sum^{{\rm N}}_{{\rm j=1}}{{\widetilde{{\rm T}}}_{{\rm m}}{\rm (}\frac{\pi {\rm (2j-1)}}{{\rm 2N}}{\rm )}}{\widetilde{{\rm T}}}_{{\rm n}}\left(\frac{\pi {\rm (2j-1)}}{{\rm 2N}}\right){\rm =}{\delta }^{{\rm m}}_{{\rm n}}{\rm ,\ 0}\le {\rm m,}{\rm n}{\rm <}{\rm N}{\rm -}{\rm 1}
\end{equation}

\vspace{0.3cm}

Умножаем каждое уравнение из \ref{eq:system} на $T_0(r_j)$, складываем их и, воспользовавшись~\ref{eq:lemma}, получаем выражение для 

\[{{\rm a}}_0{\rm =}\frac{\sum^{{\rm N}}_{{\rm j=1}}{{{\rm x}}_{{\rm j}}}}{{\rm N}}\] 
Проведя аналогичные операции с $T_i(r_j) i=\overline{1,3}$  получаем

\[{{\rm a}}_{{\rm i}}{\rm =}\frac{{\rm 2}\sum^{{\rm N}}_{{\rm j=1}}{{{\rm x}}_{{\rm j}}{{\rm T}}_{{\rm i}}{\rm (}{{\rm x}}_{{\rm j}}{\rm )}}}{{\rm N}}\] 

Итак искомые коэффициенты аппроксимирующей кривой равны

\begin{equation} \label{eq:coeff} 
\left\{ \begin{array}{c}
{{\rm a}}_0{\rm =}\frac{\sum^{{\rm N}}_{{\rm j=1}}{{{\rm x}}_{{\rm j}}}}{{\rm N}} \\ 
{{\rm a}}_{{\rm i}}{\rm =}\frac{{\rm 2}\sum^{{\rm N}}_{{\rm j=1}}{{{\rm x}}_{{\rm j}}{{\rm T}}_{{\rm i}}{\rm (}{{\rm x}}_{{\rm j}}{\rm )}}}{{\rm N}}{\rm ,\ \ }i=\overline{1,3}{\rm \ \ } \end{array}
\right. 
\end{equation} 

\subsection{Преобразование коэффициентов при переходах между базисами}

Проведя замену  $z=2t-1$, получаем 

\[\overrightarrow{B}\left(t\right)=\sum^3_{i=0}{{\tilde{T}}_i(t)\overrightarrow{a_i}},      t\in [0,1]\] 

С другой стороны 

\begin{equation} \label{eq:polinom} 
\overrightarrow{B}\left(t\right)=\sum^3_{i=0}{t^i\overrightarrow{q_i}},      t\in [0,1] 
\end{equation} 

Попарно приравнивая выражения \ref{eq:bernstein},\ref{eq:cheb} и \ref{eq:polinom}, группируя по степеням z и решая полученные системы, будем искать выражения связи для  коэффициентов в разных базисах

\[\left\{ \begin{array}{c}
a_0-a_1+a_2-a_3=p_0=q_3 \\ 
2a_1-8a_2+18a_3={-3p}_0{+3p}_1=q_2 \\ 
8a_2-48a_3={3p}_0{-6p}_1{+3p}_2=q_1 \\ 
32a_3={-p}_0{+3p}_1{-3p}_2{+p}_3=q_0 \end{array}
\right.,\] 
откуда:

\[\left\{ \begin{array}{c}
2a_0+2a_2=p_0{+p}_3 \\ 
2a_1+2a_3=p_3{-p}_0 \\ 
 \begin{array}{c}
2a_0-\frac{10}{3}a_2=p_1{+p}_2 \\ 
{\frac{2}{3}a}_1+10a_3=p_1{-p}_2 \end{array}
 \end{array}
\right.,\] 

И в итоге получаем формулы преобразования между взаимно-однозначными коэффициентами разложения уравнения кривой третьего порядка в базисе из многочленов Чебышева ($\overrightarrow{a}$), коэффициентами разложения по базису из полиномов Бернштейна ($\overrightarrow{p}$) и коэффициентами разложения той же кривой по степеням z  ($\overrightarrow{q}$)

\begin{equation} \label{eq:ChtoB} 
\overrightarrow{p}=\left( \begin{array}{c}
p_0 \\ 
p_1 \\ 
p_2 \\ 
p_3 \end{array}
\right)=\left( \begin{array}{cccc}
1 & -1 & 1 & -1 \\ 
1 & -\frac{1}{3} & -\frac{5}{3} & 5 \\ 
1 & \frac{1}{3} & -\frac{5}{3} & -5 \\ 
1 & 1 & 1 & 1 \end{array}
\right)\left( \begin{array}{c}
a_0 \\ 
a_1 \\ 
a_2 \\ 
a_3 \end{array}
\right)=A\overrightarrow{a} 
\end{equation} 

\begin{equation} \label{eq:PtoCh} 
\overrightarrow{p}=\left( \begin{array}{cccc}
1 & \frac{1}{2} & \frac{3}{8} & \frac{5}{16} \\ 
0 & \frac{1}{2} & \frac{1}{2} & \frac{15}{32} \\ 
0 & 0 & \frac{1}{8} & \frac{3}{16} \\ 
0 & 0 & 0 & \frac{1}{32} \end{array}
\right)\left( \begin{array}{c}
q_0 \\ 
q_1 \\ 
q_2 \\ 
q_3 \end{array}
\right)=B\overrightarrow{q} 
\end{equation} 

\begin{equation} \label{eq:ChtoP} 
\overrightarrow{q}=B^{-1}\overrightarrow{p}=\left( \begin{array}{cccc}
1 & -1 & 1 & -1 \\ 
0 & 2 & -8 & 18 \\ 
0 & 0 & 8 & -48 \\ 
0 & 0 & 0 & 32 \end{array}
\right)\overrightarrow{p} 
\end{equation} 

Используя формулу преобразования~\ref{eq:ChtoB} для коэффициентов, полученных по формулам~\ref{eq:coeff}, находим опорные точки кривой Безье, которая аппроксимирует элемент границы. 

\subsection{Определение параметров алгоритма}

В предыдущих пунктах мы разработали методику аппроксимации отдельного элемента границы. Однако перед нами стоит задача аппроксимации границы целиком. В связи с этим возникает вопрос: элемент какой длинны надо брать на каждом следующем шаге; или другим словами каким выбрать $M$ см~\ref{ss:Chebaprox}. Рассмотрим Рис.~\ref{ris:bob}. Верхняя часть этого рисунка довольно проста и может быть с достаточной точностью аппроксимирована 5-6 кривыми Безье, однако, нижняя часть требует гораздо большей мелкости разбиения что бы передать информацию о границе. По этой причине алгоритм должен быть адаптивным, что значит величина элементов должна меняться в соответствии с кривизной области. Кроме того стоит вопрос о выборе $N$. 

\begin{figure}[h]
\center{\includegraphics[width=0.5\linewidth]{bob}}
\caption{Фигура с сильно меняющейся кривизной границы.}
\label{ris:bob}
\end{figure}

Для решения возникших задач можно ввести функцию $q$, которая будет обозначать степень отклонения аппроксимирующей кривой $F$ от элемента который она приближает $f$, представленного в виде списка граничных точек~\ref{ss:Chebaprox}. Рассмотрим $q$, определяемую по формуле:

\begin{equation} \label{eq:qfunction} 
q(F,N,M)= \frac{\sum^N_{i=1}{\|f(t^*_{N,i})-F(t_{N,i})\|_1}}{N}
\end{equation}

Используется Манхетенская метрика: расстояние определено как сумма расстояний между координатами $\|A\|_1 := \sum_{i=1}^{n} |A_i|$, где $n$ - размерность пространства, $A_i$ координаты точки~\cite{Taxicab Geometry}. При этом Чебышевская система узлов определяется на отрезке в пределах которого изменяться параметр кривой. Параметр $M$ входит неявно, как длина списка по-сути дискретной функции $f$.

Задание таким образом функции ошибки имеет свои плюсы и минусы. Так, например, она может быть быстро вычислена, однако не очевидно точно ли она показывает ошибку, т.к. информация из не опорных точек не используется. Так же неясным являться связь $q$ и $N$. Эти проблемы будут рассмотрены подробнее далее. 

Когда выбрана функция погрешности можно приступить к распознанию границы целиком. Для этого устанавливаем некоторый интервал допустимой погрешности $q\in[q_{min},q_{max}]$. Выбираем произвольное $M$, и проводим аппроксимацию первого элемента. Вычисляем $q$ для полученной кривой, и если она больше $q_{max}$, то уменьшаем $M$, а если меньше $q_{min}$, то соответственно $M$ увеличиваем. Операцию повторять пока ошибка полученной Безье кривой не будет лежать внутри заданного интервала. Таким образом получен адаптивный алгоритм распознания границы. В программе использован измененный алгоритм. В котором устанавливаться только $q_{max}$, а увеличение $M$ реализовано иначе. Его описание заняло бы много времени. Приведенный выше алгоритм боле нагляден и полностью передает философию подхода. 

Проведем анализ эффективности алгоритма использующего такую функцию ошибки. Выбираем несколько областей, с различными характерами границ. Проведем их распознание для различных значений $q_{max}$ и $N$. Рассмотрим количество операций распознания, количество элементов разбиения и главное эстетический вид. В результате установлено несколько важных фактов:
\begin{enumerate}
\item эстетические качества результатов с одинаковыми количество элементов разбиения практически не различаются;
\item количество операций линейно связано с количество элементов разбиения;
\item качественный характер связей не меняется от области к области
\end{enumerate}

\begin{figure}
\center{\includegraphics[width=0.9 \linewidth]{NQ}}
\caption{Зависимость количества элементов разбиения от $q_{max}$ (ось$x$) и $N$ (ось$y$). Цвет меняется от темно синего - малое число элементов, до бардового - большое}
\label{ris:NQ}
\end{figure}

\begin{figure}
\center{\includegraphics[width=0.9 \linewidth]{NQsph}}
\caption{Зависимость количества элементов разбиения от $\alpha$ (ось$x$) и $R$ (ось$y$). Цвет меняется от темно синего - малое число элементов, до бардового - большое }
\label{ris:NQsph}
\end{figure}

Рассмотрим график зависимости мелкости разбиения от $q_{max}$ и $N$ (Рис.~\ref{ris:NQ}). Как видим мелкость, зависит не просто от $q_{max}$, но от обоих параметров в комплексе. Можно так же заметить что изолинии представляют из себя прямые выходящие из точки $q_{max}=0, N=0$ под разными углами. 
Проведя преобразование пространства, так что-бы изолинии стали прямыми параллельными оси $OY$ 
\[R=\sqrt{q_{max}^2+N^2}\]
\[\alpha=\arctan{\frac{q_{max}}{N}}\]
Получаем следующий Рис.~\ref{ris:NQsph}. Волны в нижнем левом углу объясняется сильными флуктуациями мелкости разбиения при изменении $N$ на маленьких значениях.

Из графика видно что целесообразно в качестве ошибки брать не просто $q$, а $\alpha=\arctg{\frac{q}{N}}$. Тогда настройка мелкости будет производится вариацией $\alpha$ на интервале $(0,\frac{\pi}{2})$. А старые параметры будут равны $q_{max}=R\sin{\alpha}$ и $N=R\cos{\alpha}$. Величину $R$ выбираем как можно меньше, однако что-бы наблюдаемые в нижней части графика флуктуации были незначительными. Из  Рис.~\ref{ris:NQsph} видно что таким критериям удовлетворяет $R=30$.

\section{Работа с Б-областями}\label{sec:methods}

Выше был разработан метод построения Б-области, однако для их применения в алгоритмах раскладки графов необходимо научиться отвечать на некоторые вопросы. В моей работе была реализована раскладка точечных вершин с учетом локализации. Во время работы алгоритма требовалось отвечать на вопросы: лежит ли данная точка внутри области, находить проекцию точки на область. При решении разных задач с использование Б-областей возникают другие вопросы. Например при раскладке вершин с заданными формами необходимо уметь определять пересекаются ли две Б-области. В этом пункте будут рассмотрены задачи возникшие при раскладке графа учетом локализации вершин.

\subsection{Вопрос о принадлежности точки области}

Воспользуемся методом трассировки луча. Этот метод заключается в том, что из точки проводиться луч в произвольном выбранном направлении. В данном случае был выбран горизонтальный луч направленный вправо. Подсчитываем количество пересечений с границей; если их четное количество то точка находится вне области, если нечетное - внутри~\cite{bezier}. Далее воспользуемся свойством кривой Безье: кривая лежит внутри выпуклой оболочки своих опорных точек. Таким образом, если луч не пересекает оболочки, он не пересекает и самой кривой.  Это свойство позволяет сразу при помощи простых проверок отметать те кривые кривые, которые заведомо не пересекаются с лучом. Рационально так-же сначала проверять пересечение луча с оболочкой всей Б-области. Если после таких проверок еще остались не откинутые элементы, находим количество пересечений аналитически при помощи формулы Кардано.

\subsection{Проецирование точки на область}

Проекцией точки на замкнутую область называют множество точек данной области на которой функционал растояния до проецируемой точки принимает минимальное значение. Будем понимать под проекцией любую из точек данного множества. Для начала также отсеем заведомо далеко лежащие элементы области. Для каждой такой Безье кривой найдем два числа: расстояние от проецируемой точки до ближайшей точки выпуклой оболочки опорных точек кривой $l_i$ и до самой отдаленной $L_i$,  где и $i$ номер кривой Безье. Найдем минимальное $L_{min} \leq L_i,  \forall i$ и исключим из рассмотрения все кривые, для которых $l_i \geq L_{min}$ Теперь необходимо найти проекцию на каждый из оставшихся элементов, и выбрать ближайшую из них. Таким образом задача сводится к нахождению проекции на кубическую Безье кривую. Запишем функционал расстояния $L=\sqrt{x(t)^2+y(t)^2}, t\in[0,1]$ и будем минимизировать его по $t$. Найти решение аналитически не возможно, по этому необходимо воспользоваться численными методами, например методом Ньютона. В программе используется более грубый метод. Кривая разбивается на равные по приращению параметра части. В точках разбиения считается $L$ решением будет та точка где эта функция минимальна.

\begin{figure}
\center{\includegraphics[width=0.6 \linewidth]{proection}}
\caption{Проекция точки на Б-область}
\label{ris:NQsph}
\end{figure}

\chapter{Применение Б-Областей в алгоритмах раскладки графов} 

Б-области могут быть использованы в алгоритмах раскладки, при этом их можно использовать для задания формы как вершин, так и локаций. Для этого необходимо производить некоторые расширения алгоритмов. Данные расширения основываются на наложении некоторых ограничений на перемещения вершин и могут легко переноситься практически на любые типы алгоритмов раскладки. По этому, целесообразно рассмотреть принципы учета этих ограничений на примере модификации одного из алгоритмов раскладок. Далее будет рассмотрена раскладка графов с учетом локализации вершин с использованием Б-областей. В качестве модифицируемого алгоритма выбран довольно популярный и дающий хорошие результаты алгоритм Gem~\cite{bilevich}. Ниже следует его короткое описание.

\section{Алгоритм Gem}

Стохастический алгоритм Gem основан на физической  аналогии и использовании ряда методик ускорения сходимости, таких как локальные температуры, гравитационные силы и определение вращений и колебаний. К тому же, стохастическая сущность алгоритма позволяет применять его для получения изображений относительно больших графов удовлетворяющих многим ограничениям и эстетическим критериям.

Температура используется в алгоритме Gem для указания максимального перемещения вершины за итерацию. Для каждой вершины локальная температура по определению зависит от предыдущего значения температуры и от наличия колебаний и вращений вместе с подграфом. Локальная температура возрастает, если вершина отдаляется от своего конечного положения. Глобальная температура является усредненным значением локальных температур всех вершин и определяет степень устойчивости изображения.

Алгоритм Gem состоит из двух основных фаз – инициализации начального состояния и фазы итераций. Инициализация состоит в задании начальных положений вершин, их импульсов и значений локальных температур.

Основной цикл на каждой итерации изменяет положение вершин до тех пор, пока значение глобальной температуры меньше предпочтительного значения минимальной температуры или до истечения разрешенного времени выполнения.

В ходе каждой итерации случайным образом выбирается подмножество вершин для изменения состояния. Далее, для каждой из выбранных вершин подсчитываются силы притяжения и отталкивания по отношению к остальным вершинам. Также подсчитывается сила притяжения к центру масс, что позволяет удерживать несвязанные компоненты графа, если они имеются. Результирующая сила масштабируется с учетом текущей температуры выбранной вершины, формируя её импульс. Масштабирование с использованием локальной температуры является главным принципом получения устойчивого расположения. Низкое значение температуры является признаком того, что расположение достаточно стабильно.

\section{Модификация алгоритма}
Что бы учесть локализацию вершины предполагается внести следующее изменение в алгоритм. Нам понадобятся методы разработанные в разделе~\ref{sec:methods} Перед совершением перемещения вершины проверять не покинет ли она после этого своей области локализации. Для этого проверяем методом трассировки луча проверяем принадлежит ли области точка куда предполагается переместить вершину. Если да то производим перенос. Если нет, то при помощи разработанных методов проецирования находим ближайшую к ней точку, принадлежащую области и переносим вершину туда. Заданное подобным образом ограничение эквивалентно заданию сил отталкивания между границей и вершиной направленной по нормали к границе, потенциал которой претерпевает скачок от нуля до бесконечности в граничной точке. Таким образом вершина свободно перемещается внутри области, однако не может покинуть ее.

\chapter{Результаты}
Полученные выше теоретические результаты были реализованы на компьютере. Для программирования использовался язык Java. Было написано несколько программ. В том числе тестовая программа распознания границ и модуль программы Esher, позволяющий учитывать локализацию вершин при раскладках графов. В данной главе приведены некоторые результаты полученные проведенных тестов, а также результаты раскладок графов на плоскости с использованием Б-областей для задания областей локализации.

\section{Сравнение производительности алгоритмов распознавания}
Проведено сравнение скорости работы разработанного алгоритма распознания с аналогичными по функциональности алгоритмами представленными в популярных графических пакетах (Photo Shop, Gimp). В качестве тестовых изображений выбраны сложные фрактальные фигуры (Рис.~\ref{ris:test}), с сильно меняющейся кривизной вдоль границы, на которых проявляется адаптивность алгоритма.  о внутренней ошибке.

\begin{figure}[h!]
\begin{minipage}[h]{0.47\linewidth}
\center{\includegraphics[width=1\linewidth]{branch}} a) \\
\end{minipage}
\hfill
\begin{minipage}[h]{0.47\linewidth}
\center{\includegraphics[width=1\linewidth]{tree}} \\b)
\end{minipage}
\vfill
\begin{minipage}[h]{0.47\linewidth}
\center{\includegraphics[width=1\linewidth]{Ftree}} c) \\
\end{minipage}
\hfill
\begin{minipage}[h]{0.47\linewidth}
\center{\includegraphics[width=1\linewidth]{largeFtree}} d) \\
\end{minipage}
\caption{Тестовые рисунки}
\label{ris:test}
\end{figure}

Время считалось при помощи секундомера с момента старта задачи до вывода результата на экран. При этом в состав результата для разработанного входит время, затрачиваемое на инициализацию приложения и распознание контура, в то время как для Photo Shop, Gimp на вход подавались распознанные границы. Результаты сравнения приведены в (Таб.~\ref{tab:time}) и гистограмме  (Рис.~\ref{ris:hist}). Эстетические качества полученных приближающих кривых одинаковы для всех трех программ. Как видно разработанный алгоритм работает в разы быстрее, при этом затрачиваемое время линейно зависит от длины контура. Для других сравниваемых программ наблюдается ярко выраженная нелинейность. Стоит так же заметить что Esher и Gimp работают тем быстрее, чем большая погрешность аппроксимации допускается, в то время как для FS затраты времени в данных случаях резко растут. FS на больших примерах не заканчивает работу, сообщая через длительный промежуток времени.

\begin{table}[h!]
\caption{Время работы алгоритмов распознания в секундах.}
\begin{center}
\begin{tabular}{|c|c|c|c|c|c|c|}
\hline
Программы& \multicolumn{2}{c|}{Gimp} & \multicolumn{2}{c|}{Photo Shop}& \multicolumn{2}{c|}{Esher}\\
\hline
Мелкость разбиения & Большая & Средняя & Большая & Средняя & Большая & Средняя\\
\hline
Рис~\ref{ris:test}.a) & 0.5 & 0.5 & 0.5 & 1 & 1.5 & 1  \\
\hline
Рис~\ref{ris:test}.b) & 7 & 5.5 & 4 & 32.5 & 4 & 3.5  \\
\hline
Рис~\ref{ris:test}.c) & 21.5 & 17 & 17 & 149 & 5.5 & 4.5  \\
\hline
Рис~\ref{ris:test}.d) & 90.5 & 70 & fail & fail & 14.5 & 13 \\
\hline
\end{tabular}
\end{center}
\label{tab:time}
\end{table}

\begin{figure}[h!]
\center{\includegraphics[width=1 \linewidth]{hist}}
\caption{Сравнение производительности алгоритмов распознания.}
\label{ris:hist}
\end{figure} 

\section{Примеры раскладок графов}

Ниже приведены результаты работы написанной программы раскладки графов. Для учета локализации вершин используются Б-области. На Рис.~\ref{ris:net} изображена схема некой характерной компьютерной сети(а- разложено вручную, b- разложено при помощи Esher). Подобные схемы часто встречаются сферах связанных с сетевыми технологиям, в том числе литературе \cite{networks}, программном обеспечении (NetCracker). Стоит заметить что это лишь простейший пример, показывающий возможности программы. Алгоритм может быть применен к графам гораздо больших размеров. На практике часто возникают задачи с сотнями и тысячами вершин, раскладка которых вручную занимает много времени, а порой просто невозможна. 

\begin{figure}[h!]
\begin{minipage}[h]{1.\linewidth}
\center{\includegraphics[width=1\linewidth]{net1}} a) \\
\end{minipage}
\hfill
\begin{minipage}[h]{1.\linewidth}
\center{\includegraphics[width=1\linewidth]{net2}} \\b)
\end{minipage}
\caption{Пример раскладки алгоритмом компьютерной сети. a) граф разложен вручную. b) результат работы программы}
\label{ris:net}
\end{figure}

Рис.~\ref{ris:euro} демонстрирует как программа справилась с раскладкой схемы электрических сетей стран Балканского полуострова, приведенной во вступительной части работы.

\begin{figure}[h!]
\center{\includegraphics[width=1 \linewidth]{euro}}
\caption{Результат работы алгоритма на модели электрических сетей.}
\label{ris:euro}
\end{figure} 

\chapter*{Выводы}
\addcontentsline{toc}{chapter}{Выводы}
Поставленные задачи полностью решены. Достигнуты следующие результаты:
\begin{itemize}
\item создан новый алгоритм аппроксимации растровой границы векторными кривыми Безье третьего порядка;
\item проведенное сравнение его с аналогами, представленными в популярных программах Photo Shop, Gimp, показало значительное преимущество нового алгоритма;
\item реализованы простейшие алгоритмы работы Б-областями, задающими вид произвольных, в общем случае не выпуклых, замкнутых, многосвязных областей;
\item на основе Б-областей проведена модификация алгоритмов раскладки графов, позволяющая учитывать локализацию вершин в областях сложной формы;
\item разработанные алгоритмы успешно реализованы и внедрены в алгоритм раскладки графов GEM на платформе графопостроителя Esher;
\item получены результаты раскладки различных графов, демонстрирующие эффективность предложенных методик. 
\end{itemize}

Существует два перспективных направления дальнейшей работы:
\begin{itemize}
\item построение сеток для разностных схем при помощи алгоритмов раскладки графов с учетом локализации вершин
\item перенесение результатов аппроксимации границ объектов на 3-х мерный случай
\end{itemize}


\begin{thebibliography}{99}
\addcontentsline{toc}{chapter}{Литература}

\bibitem {graphdrawing}Battista, Giuseppe Di; Eades, Peter; Tamassia, Roberto; Tollis, Ioannis G. , Graph Drawing: Algorithms for the Visualization of Graphs. 1998
\bibitem {bezier}Роджерс Д., Адамс Дж. Математические основы машинной графики. — М.: Мир, 2001.
\bibitem {gdea} Graph Drawing E-print Archive, http://gdea.informatik.uni-koeln.de/ University of Southampton. 2009
\bibitem {chislmetod} Бахвалов Н. С., Жидков Н. П., Кобельков Г. М., Численные методы. — М.: Наука, 2003.
\bibitem {softwarelist} Open Directory Project, http://www.dmoz.org/ Science/Math/Combinatorics/Software. 2009
\bibitem {shokurov}Шокуров А.В. Теоретические основы методов вычислений.Конспект лекций. М., ИСП РАН. 2008 
\bibitem {numrical analysis}Stewart, Gilbert W. (1996), Afternotes on Numerical Analysis, SIAM, ISBN 978-0-89871-362-6.
\bibitem {Taxicab Geometry} Eugene F. Krause (1987). Taxicab Geometry. Dover. ISBN 0-486-25202-7. 
\bibitem {bilevich} Белевич П., Стохастический алгоритм раскладки неориентированных графов на плоскости с учетом кластеризации, бакалаврский диплом. МФТИ, 2007
\bibitem {networks}Олифер В.Г, Олифер Н.А Компьютерные Сети. 3-е издание, Питер. 2007.

\end{thebibliography}

\end{document}
